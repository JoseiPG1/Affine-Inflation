\documentclass[10pt,a4paper]{article}
\usepackage[utf8]{inputenc}
\usepackage{amsmath}
\usepackage{amsfonts}
\usepackage{amssymb}

\usepackage{vmargin}

\usepackage{slashed}


\title{Affine Inflation in Polynomial Affine Gravity in $3+1$ dimensions}
\author{Jose Perdiguero Garate}
\date{20/12/2022}			

\begin{document}

\maketitle

\begin{abstract}
  The Polynomial Affine Gravity its a purely affine model that mediates gravitational interactions solely and exclusive through the
  affine connection instead of the metric tensor. In this paper we couple a scalar field through \textit{inverse tensor densities} and 
  its potential energy to the volume form. We formulte an effective action in the torsion-free sector couple with the scalar field and
  study the cosmological solutions.
\end{abstract}

\tableofcontents

\section{Introduction}

General Relativity, being so far the most successful model of gravitational interactions, is the ground basis of the standard model 
of cosmology, also known as $\Lambda$CDM. However, the later rely in a number of additional ingredients to describe consistently the evolution
of the cosmos in a way that is compatible with the observations. It should be mentioned that the observations come from different ages of the Universe:
cosmic microwave background, big bang nucleo-synthesis and baryon acoustic oscillations: and there are also additional observations coming from 
gravitational lenses and gravitational waves.

Nonetheless, at any early stage of the Universe it is need an inflationary epoch with the purpose of solving the flatness and horizon problems. The
inflation is generally induced by a scalar field which couples to the gravitiational sector and possesses a self-interacting term that
determines the prices conditions of the model.

\section{Polynomial Affine Gravity}

The Polynomial Affine Gravity model its a purely affine model on which we endowed the manifold only with an affine connection 
$(\mathcal{M}, \Gamma)$. This allow us to define the notion of parallelism by the covariant derivative $\nabla$. Since we only
have an affine connection $\Gamma$ we can only deffine the following chain of geometric objects

\begin{equation}
  \Gamma_{\mu}{}^{\sigma}{}_{\nu} \to
  \nabla_\mu \to \mathcal{R}_{\mu\sigma}{}^{\tau}{}_{\nu} \to \mathcal{R}_{\mu\nu}  
\end{equation} 

Notice that in the absence of the metric tensor it is not possible to define the $\mathcal{R}$.

\subsection{The action}

In order to built the action of the Polynomial Affine Gravity we use the irreducible fields of the affine connection,
by separating the connection into its symmetric and antisymmetric part

\begin{equation}
  \hat{\Gamma}_{\mu}{}^{\sigma}{}_{\nu} = \Gamma_{\mu}{}^{\sigma}{}_{\nu} + \mathcal{B}_{\mu}{}^{\sigma}{}_{\nu} + \delta^{\sigma}_{[\mu}\mathcal{A}_{\nu]}
\end{equation}

where $\Gamma_{\mu}{}^{\sigma}{}_{\nu}$ correspond to the symmetric part of the connection, $\mathcal{B}_{\mu}{}^{\sigma}{}_{\nu}$ its the traceless
part of the torsion tensor and $\mathcal{A}_{\mu}$ its the vectorial part of the torsion tensor. Additionally, we need to define the volume
form, which can be written using only the wedge product

\begin{equation}
  \mathrm{d}V^{\alpha\beta\gamma\delta} = J(x)\mathrm{d}x^{\alpha}\wedge\mathrm{d}x^{\beta}\wedge\mathrm{d}x^{\gamma}\wedge\mathrm{d}x^{\delta}
\end{equation}

However, since we want to couple a scalar field $\phi (x)$ to this model, we need to introduce the potential energy of the 
scalar field $\mathcal{V}(\phi)$. Inspired by the work of Hemza Azri in affine inflation, we couple the potential energy to
the volume form in the following manner

\begin{equation}
  \mathrm{d}V^{\alpha\beta\gamma\delta} = \mathrm{d}\hat{V}^{\alpha\beta\gamma\delta}\frac{1}{\mathcal{V}(\phi)}
\end{equation}

The action must preserv the invariance under diffemorphism, which is why the symmetric part of the connection goes indirectly throught the 
covariant derivative. The fundamental fields to build the action are $\nabla, \mathcal{A}, \mathcal{B}, \mathrm{d}V$. Then we perform a sort
of \textit{dimensional structural analysis technique} studying everysingle possible non-trivial contribution to the action.

THen, the most general action in $3+1$ dimension up to boundary terms is
\begin{equation*}
    \begin{split}
      S
      & =
      \int  \mathrm{d}V^{\alpha \beta \gamma \delta} \bigg[
      B_1 \mathcal{R}_{\mu\nu}{}^{\mu}{}_{\rho}\mathcal{B}_{\alpha}{}^{\nu}{}_{\beta}\mathcal{B}_{\gamma}{}^{\rho}{}_{\delta}
      + B_2 \mathcal{R}_{\alpha\beta}{}^{\mu}{}_{\rho} \mathcal{B}_{\gamma}{}^{\nu}{}_{\delta} \mathcal{B}_{\mu}{}^{\rho}{}_{\nu}
      + B_3 \mathcal{R}_{\mu\nu}{}^{\mu}{}_{\alpha} \mathcal{B}_{\beta}{}^{\nu}{}_{\gamma} \mathcal{A}_\delta
      + B_4 \mathcal{R}_{\alpha\beta}{}^{\sigma}{}_{\rho}\mathcal{B}_{\gamma}{}^{\rho}{}_{\delta}\mathcal{A}_\sigma
      \\
      & \quad
      + B_5 \mathcal{R}_{\alpha \beta}{}^{\rho}{}_{\rho} \mathcal{B}_{\gamma}{}^{\sigma}{}_{\delta} \mathcal{A}_\sigma
      + C_1 \mathcal{R}_{\mu\alpha}{}^{\mu}{}_{\nu} \nabla_\beta \mathcal{B}_{\gamma}{}^{\nu}{}_{\delta}
      + C_2 \mathcal{R}_{\alpha\beta}{}^{\rho}{}_{\rho} \nabla_\sigma \mathcal{B}_{\gamma}{}^{\sigma}{}_{\delta}
      + D_1 \mathcal{B}_{\nu}{}^{\mu}{}_{\lambda} \mathcal{B}_{\mu}{}^{\nu}{}_{\alpha} \nabla_\beta \mathcal{R}_{\gamma}{}^{\lambda}{}_{\delta}
      \\
      & \quad
      + D_2 \mathcal{B}_{\alpha}{}^{\mu}{}_{\beta} \mathcal{B}_{\mu}{}^{\lambda}{}_{\nu} \nabla_{\lambda} \mathcal{B}_{\gamma}{}^{\nu}{}_{\delta}
      + D_3 \mathcal{B}_{\alpha}{}^{\mu}{}_{\nu}\mathcal{B}_{\beta}{}^{\lambda}{}_{\gamma} \nabla_\lambda \mathcal{B}_{\mu}{}^{\nu}{}_{\delta}
      + D_4 \mathcal{B}_{\alpha}{}^{\lambda}{}_{\beta}\mathcal{B}_{\gamma}{}^{\sigma}{}_{\delta}\nabla_\lambda \mathcal{A}_\sigma
      + D_5 \mathcal{B}_{\alpha}{}^{\lambda}{}_{\beta} \mathcal{A}_\sigma \nabla_\lambda \mathcal{B}_{\gamma}{}^{\sigma}{}_{\delta}
      \\
      & \quad
      + D_6 \mathcal{B}_{\alpha}{}^{\lambda}{}_{\beta}\mathcal{A}_\gamma \nabla_\lambda A_\delta
      + D_7\mathcal{B}_{\alpha}{}^{\lambda}{}_{\beta} \mathcal{A}_\lambda \nabla_\gamma A_\delta
      + E_1\nabla_\rho \mathcal{B}_{\alpha}{}^{\rho}{}_{\beta} \nabla_\sigma \mathcal{B}_{\gamma}{}^{\sigma}{}_{\delta}
      + E_2 \nabla_\rho \mathcal{B}_{\alpha}{}^{\rho}{}_{\beta} \nabla_\gamma \mathcal{A}_\delta
      \\
      &\quad
      + F_1 \mathcal{B}_{\alpha}{}^{\mu}{}_{\beta} \mathcal{B}_{\gamma}{}^{\sigma}{}_{\delta} \mathcal{B}_{\mu}{}^{\lambda}{}_{\rho} \mathcal{B}_{\sigma}{}^{\rho}{}_{\lambda}
      + F_2\mathcal{B}_{\alpha}{}^{\mu}{}_{\beta} \mathcal{B}_{\gamma}{}^{\nu}{}_{\lambda} \mathcal{B}_{\delta}{}^{\lambda}{}_{\rho} \mathcal{B}_{\mu}{}^{\rho}{}_{\nu}
      + F_3 \mathcal{B}_{\nu}{}^{\mu}{}_{\lambda} \mathcal{B}_{\mu}{}^{\nu}{}_{\alpha} \mathcal{B}_{\beta}{}^{\lambda}{}_{\gamma} \mathcal{A}_\delta
      + F_4 \mathcal{B}_{\alpha}{}^{\mu}{}_{\beta}\mathcal{B}_{\gamma}{}^{\nu}{}_{\delta}\mathcal{A}_\mu \mathcal{A}_\nu \bigg].
    \end{split}
  \end{equation*}

In previous works we have mentioned some of the features of the action: (i) Its rigidity, since contains all possible combinations of the fields and their derivatives; 
(ii) All the coupling constants are dimensionless, which might be a sign of conformal invariance, and also ensure that the model is power-counting renormalisable; 
(iii) The field equations are second order differential equations, and the Einstein spaces are a subset of their solu- tions; 
(iv) The supporting symmetry group is the group of diffeomorphisms, desirable for the background independence of the model; 
(v) Even though there is no fundamental metric, it is possible to obtain emergent (connection- descendent) metric tensors; 
(vi) The cosmological constant appears in the solutions as an integration constant, changing the paradigm concerning its interpretation; 
(vii) The model can be extended to be coupled with a scalar field, and the field equations are equivalent to those of General Relativity interacting with a massless scalar field. 



To couple the affine action to a scalar field, we need to introduce a kinetic term in the absence of the metric tensor. In order to do so, we build 
\textit{inverse symmetric tensor densities}, by using the \textit{dimensional analysis structure technique}

\begin{equation}
      \mathrm{g}^{\mu\nu} = \left(\alpha \nabla_\lambda \mathcal{B}_{\rho}{ }^{\mu}{ }_{\sigma} + \beta \mathcal{A}_\lambda 
      \mathcal{B}_{\rho}{ }^{\mu}{ }_{\sigma}\right)\mathrm{d}V^{\nu\lambda\rho\sigma} + \gamma \mathcal{B}_{\kappa}{ }^{\mu}{ }_{\lambda}
      \mathcal{B}_{\rho}{ }^{\nu}{ }_{\sigma}\mathrm{d}V^{\kappa\lambda\rho\sigma}
\end{equation}

Using the above expression we can define the kinetic term 

\begin{equation}
    S_{\phi} = - \int \mathrm{g}^{\mu\nu} \partial_{\mu} \phi \partial_{\nu} \phi
\end{equation}

Since we want to work on the torsion-free sector, it is worth to notice that only the terms that are linear in the torsion will
have a non-trivial contribution, $C_1$ and $C_2$. Additionally, since our connection its an \textit{equi-affine} connection, the trace of the
Riemman tensor will vanish completly. Applying the same idea the to the scalar field action, only the $\alpha$ term survive. Thus,
the effective action coupled with a scalar field is

\begin{equation*}
    \begin{split}
      S_{ef}
      & =
      \int  \mathrm{d}V^{\alpha \beta \gamma \delta} \bigg[
      C_1 \mathcal{R}_{\mu\alpha}{}^{\mu}{}_{\nu} - \alpha \partial_{\alpha}\phi \partial_{\nu}
      \bigg]\nabla_\beta \mathcal{B}_{\gamma}{}^{\nu}{}_{\delta}.
    \end{split}
  \end{equation*}


\subsection{Cosmological ansatz}

In order to solve the field equations, one need to build an ansatz, since we want to do cosmology, we need to build an ansazts
compatible with the symmetries of the cosmological principle, which are rotation and translations. It is possible to build an ansatz for our
fundamental geometric objects using the Lie derivative along the Killing vector fields. The most general ansatz for the symmetric part of the 
connection $\Gamma_{\mu}{}^{\sigma}{}_{\nu}$ is
\begin{align}
      \Gamma_{t}{}^{t}{ }_{t} & =f(t), \quad \Gamma_{i}{ }^{t}{ }_{j}=g(t) S_{i j} \\
      \Gamma_{t}{ }^{i}{ }_{j} &= h(t) \delta^{i}_{j}, \quad \Gamma_{i}{ }^{j}{ }_{k}= \gamma_{i}{ }^{j}{ }_{k}
\end{align}
  
the traceless part of the torsion tensor $\mathcal{B}_{\mu}{}^{\sigma}{}_{\nu}$ is completely define by only one time depending function
  \begin{align*}
      \mathcal{B}_{\theta}{ }^{r}{ }_{\varphi} & = \psi (t) r^2\sin\theta \sqrt{1 - \kappa r^2} &
      \mathcal{B}_{r}{}^{\theta}{}_{\varphi} & =\frac{\psi (t) \sin \theta}{\sqrt{1 - \kappa r^2}} \\
      \mathcal{B}_{r}{}^{\varphi}{}_{\theta} & =\frac{\psi(t)}{ \sqrt{1-\kappa r^{2}} \sin \theta}
  \end{align*}
  
and finally, the vectorial torsion tensor $\mathcal{A}_\mu$ is given by
  \begin{align}
      \mathcal{A}_{t} = \eta(t)
  \end{align}

Since, we have the tensor $g_{\mu\nu}$ presented as a particular solution to the field equation. Thus, we need to provied
an ansatz for this tensor compatible with the symmetries of the cosmological principle
\begin{equation}
  g_{\mu\nu} = b(t)\mathrm{d}t^2 + a(t)\left(\frac{1}{1 - \kappa r^2} \mathrm{d}r^2 + r^2 \mathrm{d}\theta^2 
  + r^2 \sin^2 \theta \mathrm{d} \varphi^2 \right)
\end{equation}

Additionally, its required that the covariant derivative of the object $g_{\mu\nu}$ must vanishes completly, from which we found that
\begin{align}
  b(t) & = -b_0 & h(t) & = \frac{\dot{a}(t)}{a(t)} & g(t) & = \frac{\dot{a}(t)a(t)}{b_0}
\end{align}
It is important to remark to the above definitions will only have an effect on the parallel field equation, which 
is where the $g_{\mu\nu}$ object exist.


\subsection{The field equations}

The field equations are obtained using Kiwosjki's formalism and taking into account the symmetries and properties of the fundamental fields, we 
vary the action with respect to the fundamental fields. In the torsion-free limit, the field equation is

\begin{equation}
  \nabla_\mu \biggl[\frac{1}{\mathcal{V}(\phi)} \left(C \partial_\alpha \phi \partial_\lambda \phi - \mathcal{R}_{\alpha\lambda}\right)\mathrm{d}V^{\mu\nu\rho\alpha}\biggr] 
  + \frac{2}{3}\nabla_\mu \biggl[ \frac{1}{\mathcal{V}(\phi)}\mathcal{R}_{\alpha\theta} \delta^{[\nu}_{\lambda}\mathrm{d}V^{\rho]\alpha\mu\theta} \biggr] = 0
\end{equation}

By multiplying the left hand side of the field equation by $\epsilon_{\nu\rho\tau\beta}$, the second term vanishes completly and the field 
equation is reduced even further to

\begin{equation}
  \nabla_{[\mu}\left(\mathcal{R}_{\nu]\gamma}\frac{1}{\mathcal{V}(\phi)}\right) 
  - C \nabla_{[\mu}\left(\partial_{\nu]} \phi \partial_\gamma \phi \frac{1}{\mathcal{V}(\phi)}\right) = 0
\end{equation}

A particular solution to the above equation is

\begin{equation}
  \mathcal{R}_{\mu\nu} - C\partial_\mu \phi \partial_\nu \phi = \Lambda \mathcal{V}(\phi) g_{\mu\nu}
\end{equation}

which can be written as
\begin{equation}
  \mathcal{R}_{\mu\nu} - \frac{1}{2}\mathcal{R}g_{\mu\nu} + \Lambda \mathcal{V}(\phi) g_{\mu\nu} = 
  C\left(\partial_\mu \phi \partial_\nu \phi - \frac{1}{2}g_{\mu\nu} (\partial \phi)^2\right)
\end{equation}
Taking the divergence $\nabla^\mu$ of the above equation leads to
\begin{equation}
  C \nabla^\mu \nabla_\mu \phi - \Lambda \mathcal{V}'(\phi) = 0
\end{equation}
This is the Klein-Gordon field equation which requires the existence of the $g_{\mu\nu}$ object, to define the 
d'Alembert operator. Without this object, it is not possible to obtain a Klein-Gordon field equation, 
and additionally we require that the integration constant $\Lambda \neq 0$.

From the field equation obtained by varying the action, we distinguish two families of solutions, the first one being the
\textit{Reduced field equation} couple with the Klein-Gordon field equation
\begin{align}
  \frac{\mathcal{R}_{\mu\nu} - C\partial_\mu \phi \partial_\nu \phi}{\mathcal{V}(\phi)} & = \Lambda g_{\mu\nu} & 
  C \nabla^\mu \nabla_\mu \phi - \Lambda \mathcal{V}'(\phi) & = 0
\end{align}
where the Klein-Gordon field equation emerge naturally from the parallelism constraint. The second type of solution its given by
\textit{Harmonic field equation} which is coupled with a kinetic term and potential energy
\begin{equation}
  \nabla_{[\mu}\left(\mathcal{R}_{\nu]\gamma}\frac{1}{\mathcal{V}(\phi)}\right) 
  - C \nabla_{[\mu}\left(\partial_{\nu]} \phi \partial_\gamma \phi \frac{1}{\mathcal{V}(\phi)}\right) = 0
\end{equation}





\section{Cosmological Solutions}

Here we study all the possible solutions to the field equations under the simmetries of the cosmological principle. Notice that, the 
Ricci tensor can be use as an emergent metric tensor only when its well defined and not degenerate, meaning its temporal and spatial
part are not trivial. Under the cosmological ansatz the Ricci tensor is 
\begin{align}
  \mathcal{R}_{tt} & = -3\left(\dot{h}(t) + h^2(t)\right) & \mathcal{R}_{rr} & = \frac{\dot{g}(t) + g(t)h(t) + 2\kappa}{1 - \kappa r^2}
\end{align}
Notice that the expression $\dot{g}(t) + g(t)h(t) + 2\kappa$ its the affine analogue to the well-known scale factor $a(t)$ in the FRW universe. As 
one does in \textit{General Relativity}, demanding that its covariant derivative must vanishes completly, therefore, we are able 
to completely define the affine functions of the symmetric part of the connection
\begin{align}
  h(t) & = \sqrt{A_1}\tanh\left(t\sqrt{A_1}\right) & g(t) & = \frac{\kappa \sinh\left(2t\sqrt{A_1}\right) }{2\sqrt{A_1}}
\end{align}
The above definitions ensures that $\nabla_\alpha \mathcal{R}_{\beta\gamma} = 0$, this asservs the metricity condition. Then, the Ricci tensor
is written as
\begin{align}
  \mathcal{R}_{tt} & = -3A_1 & \mathcal{R}_{rr} & =  \frac{3\kappa \cosh^2(t\sqrt{A_1})}{1 - \kappa r^2}
\end{align}
Here we have an \textit{affine scale factor} given by 
\begin{equation}
  a_f(t) = \left(3\kappa \cosh^2(t\sqrt{A_1})\right)^{1/2}
\end{equation}

\subsection{Reduced equations}

Under the cosmological ansatz the field equation for the geometric part is written as follow
\begin{align}
  0 & = 3\ddot{a}(t) + C a(t)\dot{\phi}^{2}(t) - \Lambda b_0 a(t)\mathcal{V}(\phi) \\
  0 & = \ddot{a}(t)a(t) + 2\dot{a}^2(t) + 2b_0\kappa - \Lambda b_0 a^2(t)\mathcal{V}(\phi)
  \end{align}
Combining the two equations we obtained
\begin{equation}
  3H^2(t) = \Lambda b_0 \mathcal{V}(\phi) -\frac{1}{2}C \dot{\phi}^2(t) - \frac{3b_0\kappa}{a^2(t)}
\end{equation}
where the first function is the Hubble function $H(t)$. Notice that for $\kappa = 0$ we recover the classical Friedmann equation. Additionally,
the field equation for the scalar is given by
\begin{equation}
  \ddot{\phi}(t) + 3h(t)\dot{\phi}(t) + \Lambda b_0 \mathcal{V}'(\phi) = 0
\end{equation}
We are able to recover Einstein-Hillbert coupled with a scalar field, therefore, there is no new information in this type of solution.


\subsection{Harmonic equations}

The \textit{Harmonic field equation} coupled with the scalar field is given by
\begin{equation}
  \nabla_{[\mu}\left(\mathcal{R}_{\nu]\gamma}\frac{1}{\mathcal{V}(\phi)}\right) 
  - C \nabla_{[\mu}\left(\partial_{\nu]} \phi \partial_\gamma \phi \frac{1}{\mathcal{V}(\phi)}\right) = 0
\end{equation}
which can be rewritten as
\begin{equation}
  \nabla_{[\mu}\mathcal{S}_{\nu]\gamma} = 0
\end{equation}
where the $\mathcal{S}_{\mu\nu}$ tensor is a symmetric tensor defined by
\begin{equation}
  \mathcal{S}_{\mu\nu} = \left(\mathcal{R}_{\mu\nu} - C \partial_{\mu} \phi \partial_\nu \phi \right)\frac{1}{\mathcal{V}(\phi)}
\end{equation}
The \textit{Harmonic field equation} under the symmetries of the cosmological ansatz can be written as follow
\begin{equation}
  C\mathcal{V}(\phi)g\dot{\phi}^2 + \mathcal{V}(\phi)\left(4gh^2 +2\kappa h + 2g\dot{h} - \ddot{g}\right) +
  \mathcal{V}'(\phi)\dot{\phi}\left(gh + 2\kappa + \dot{g}\right) = 0
\end{equation}
Since we have two unknown functions of time and the scalar field, it is not possible to solve the above equation. However, if the Ricci tensor
it is not degenerate, then it can serve the function as a metric tensor. Therefore, demanding that is covariant derivative must be zero, 
the above equation is reduced to
\begin{equation} 
  \dot{\phi}(t) + \frac{3\sqrt{A_1}}{C}\tanh\left(t\sqrt{A_1}\right)\left(\frac{\mathcal{V}'(\phi)}{\mathcal{V}(\phi)}\right) = 0
\end{equation}
At this point we can proceed as one usually does in classical cosmology, given a potential you find the scale factor and the scalar field, whereas 
here given a potential, you only need to determined the scalar field. The most well known potentials are the \textit{Power-Law potential} and 
\textit{Starobinsky potential}.

Taking first the \textit{Power-Law potential} meaning $\mathcal{V}(\phi) = \beta \phi^n(t)$, then the above equation leads as
\begin{equation}
  \dot{\phi}(t) + \frac{3\sqrt{A_1}}{C}\tanh\left(t\sqrt{A_1}\right)\left(\phi^{-1}(t)n\right) = 0
\end{equation}
which can be solved analytically
\begin{equation}
  \phi(t) = \pm \sqrt{\frac{2C\phi_0 - 6n\log(\cosh(t\sqrt{A_1}))}{C} }
\end{equation}

Then we take \textit{Starobinsky potential} written as $\mathcal{V}(\phi) =  \alpha \left(1 - e^{-\beta\phi}\right)^2$, the equation takes
the form of
\begin{equation}
  \dot{\phi}(t) - \frac{6\beta\sqrt{A_1}}{C}\tanh\left(t\sqrt{A_1}\right)\left(\frac{1}{\left(1 - e^{-\beta\phi}\right)}\right) = 0
\end{equation}
which can be solved analytically in terms of the inverse hypergeometric functions.

Notice that for the above well-known potential in the cosmology literature, in our formulation it is a condition that $\kappa \neq 0$ in order
to have non-trivial field equation, whereas in FRW coupled with a scalar field, its a constraint that $\kappa = 0$, these is a strong difference
between our geometries. Additionally, while in FRW one needs to impose the \textit{slow-roll} conditions to solve the field equations, here
in the absence of those approximation we are able to solve the field equations.

Another approuch to solve the harmonic field equation, its to solve its simplified version menaning the affine reduced scenario $\mathcal{S}_{\mu\nu} = 0$
or the affine parallel equation $\nabla_\gamma\mathcal{S}_{\mu\nu} = 0$. Let us consider first, the reduced case. In this set up, we have two equations
\begin{align}
  \dot{h}(t) + h^2(t) + C\dot{\phi}^2(t) & = 0 & \dot{g}(t) + g(t)h(t) + 2\kappa & = 0
\end{align}
where the potential energy $\mathcal{V}(\phi)$ does not play any role. Here we have two unknown functions of time and the scalar field, therefore, we 
cannot solve the system. However, we demanding that the Ricci tensor is covariantly constant, the above equation is reduced to
\begin{align}
  C\dot{\phi}^2(t) + A_1 & = 0 & 3\kappa \cosh(t\sqrt{A_1}) & = 0
\end{align}
from the second equation its clear to see that $\kappa = 0$ must vanishes completely and from the first equation we can determined
the scalar field
\begin{equation}
  \phi(t) = \pm\sqrt{-\frac{A_1}{C}}t + \phi_0
\end{equation}
This type of solution its not usefull because the scalar field $\phi(t)$ its a linear increase function of time, meaning that the inflation
will never stop. 

The second case its the affine parallel scenario. Here we have three equations
\begin{align}
  0 & = C\mathcal{V}'\dot{\phi}^3-2C\mathcal{V}\dot{\phi}\ddot{\phi} - 3\mathcal{V}\left(2h\dot{h} + \ddot{h}\right) + 3\mathcal{V}'\dot{\phi}
  \left(h^{2} + \dot{h}\right) \\
  0 & = Cg\dot{\phi}^2 + 2gh^2 - 2\kappa h - h\dot{g} + 3g\dot{h} \\
  0 & = \left(2gh^2 + 4\kappa h +h\dot{g} - g \dot{h} - \ddot{g}\right)\mathcal{V} + \left(gh + 2\kappa + \dot{g}\right)\mathcal{V}'\dot{\phi}
\end{align}
Demanding that the Ricci tensor covariant derivative vanishes, the equations reduced to
\begin{align}
  0 & = C\mathcal{V}'\dot{\phi}^3 - 2C\mathcal{V}\dot{\phi}\ddot{\phi} + 3A_1 \mathcal{V}'\dot{\phi}\\
  0 & = C\kappa \cosh(t\sqrt{A_1})\sinh(t\sqrt{A_1})\dot{\phi} \\
  0 & = \kappa \cosh(t\sqrt{A_1})\mathcal{V}'\dot{\phi}
\end{align}
the last two equations can be solve simultaneously by fixing $\kappa = 0$, additionally, the first equation can be rewritten as
\begin{equation}
  \ddot{\phi} - \frac{1}{2}\dot{\phi}^2\left(\frac{\mathcal{V}'}{\mathcal{V}}\right) + \frac{3A_1}{2C}\left(\frac{\mathcal{V}'}{\mathcal{V}}\right) = 0
\end{equation}
Here, just like before we can give the potential energy term $\mathcal{V}(\phi)$, taking the potential \textit{Power-Law} the scalar field
can be solved in terms of inverse hypergeometric functions, but for $n = 0 , 1 , 2$ we have an analytically expression. In the case of Starobinsky
potential, the scalar field is written in terms of inverse hyperbolic tangent function. Notice that in this situation, we need to constraint $\kappa = 0$ to have non-trivial field equation, this, in some sense matches what
happens with the $\kappa$ factor in FRW coupled with a scalarfield. 

One could go even further and try to use the \textit{slow-roll} conditions, meaning that the kinetic energy of the scalar field its 
neglectable with respect to the potential energy
\begin{align}
  \mathcal{V}(\phi) & >> \frac{1}{2}\dot{\phi}^2 & \mathcal{V}'(\phi) & >> \ddot{\phi}
\end{align}
the second constraint was obtained from the first condition. Applying the above approximation to the field equations
\begin{equation}
  \left(\frac{\mathcal{V}'}{\mathcal{V}}\right)\left(\frac{3A_1}{2C\dot{\phi}^2} -1 \right)= 0
\end{equation}
from here we have to type of solutions, the first one requires that the potential energy term $\mathcal{V}(\phi)$ must be a constant. The second
type of solution requires that the scalar field its a linear function of time (just like in the affine reduced case), in this context, the scalar field its explicitly determined
as a function of time, whereas the potential energy term remains undetermined. 

We could try to solve the parallel equations by considering another well-known approximation, in FRW coupled with a scalar field, in order to ensure
to have an exponential growth of the scale factor, the hubble parameter must be a constant. In our affine formulation, this approximation its translate to demanding
that the affine function $h(t) = h_0$ must be a constant. This type of approuch lead to a system of differential equations that can be solved and additionally,
in this solution the Ricci tensor it is not covariantly constant, however it is degenerate, meaning that the spatial part is trivial. 



\section{Conclusions}

\end{document}