\documentclass[10pt,a4paper]{article}
\usepackage[utf8]{inputenc}
\usepackage{amsmath}
\usepackage{amsfonts}
\usepackage{amssymb}

\usepackage{vmargin}

\usepackage{slashed}


\title{Affine Inflation in Polynomial Affine Gravity in $3+1$ dimensions}
\author{Jose Perdiguero Garate}
\date{20/12/2022}			

\begin{document}

\maketitle

\begin{abstract}
  The Polynomial Affine Gravity its a purely affine model that mediates gravitational interactions solely and exclusive through the
  affine connection instead of the metric tensor. In this paper we couple a scalar field through \textit{inverse tensor densities} and 
  its potential energy to the volume form. We formulte an effective action in the torsion-free sector couple with the scalar field and
  study the cosmological solutions.
\end{abstract}

\tableofcontents

\section{Introduction}

General Relativity, being so far the most successful model of gravitational interactions, is the ground basis of the standard model 
of cosmology, also known as $\Lambda$CDM. However, the later rely in a number of additional ingredients to describe consistently the evolution
of the cosmos in a way that is compatible with the observations. It should be mentioned that the observations come from different ages of the Universe:
cosmic microwave background, big bang nucleo-synthesis and baryon acoustic oscillations: and there are also additional observations coming from 
gravitational lenses and gravitational waves.

Nonetheless, at any early stage of the Universe it is need an inflationary epoch with the purpose of solving the flatness and horizon problems. The
inflation is generally induced by a scalar field which couples to the gravitiational sector and possesses a self-interacting term that
determines the prices conditions of the model.

\section{Polynomial Affine Gravity}

The Polynomial Affine Gravity model its a purely affine model on which we endowed the manifold only with an affine connection 
$(\mathcal{M}, \Gamma)$. This allow us to define the notion of parallelism by the covariant derivative $\nabla$. Since we only
have an affine connection $\Gamma$ we can only deffine the following chain of geometric objects

\begin{equation}
  \Gamma_{\mu}{}^{\sigma}{}_{\nu} \to
  \nabla_\mu \to \mathcal{R}_{\mu\sigma}{}^{\tau}{}_{\nu} \to \mathcal{R}_{\mu\nu}  
\end{equation} 

Notice that in the absence of the metric tensor it is not possible to define the $\mathcal{R}$.

\subsection{The action}

In order to built the action of the Polynomial Affine Gravity we use the irreducible fields of the affine connection,
by separating the connection into its symmetric and antisymmetric part

\begin{equation}
  \hat{\Gamma}_{\mu}{}^{\sigma}{}_{\nu} = \Gamma_{\mu}{}^{\sigma}{}_{\nu} + \mathcal{B}_{\mu}{}^{\sigma}{}_{\nu} + \delta^{\sigma}_{[\mu}\mathcal{A}_{\nu]}
\end{equation}

where $\Gamma_{\mu}{}^{\sigma}{}_{\nu}$ correspond to the symmetric part of the connection, $\mathcal{B}_{\mu}{}^{\sigma}{}_{\nu}$ its the traceless
part of the torsion tensor and $\mathcal{A}_{\mu}$ its the vectorial part of the torsion tensor. Additionally, we need to define the volume
form, which can be written using only the wedge product

\begin{equation}
  \mathrm{d}V^{\alpha\beta\gamma\delta} = J(x)\mathrm{d}x^{\alpha}\wedge\mathrm{d}x^{\beta}\wedge\mathrm{d}x^{\gamma}\wedge\mathrm{d}x^{\delta}
\end{equation}

However, since we want to couple a scalar field $\phi (x)$ to this model, we need to introduce the potential energy of the 
scalar field $\mathcal{V}(\phi)$. Inspired by the work of Hemza Azri in affine inflation, we couple the potential energy to
the volume form in the following manner

\begin{equation}
  \mathrm{d}V^{\alpha\beta\gamma\delta} = \mathrm{d}\hat{V}^{\alpha\beta\gamma\delta}\frac{1}{\mathcal{V}(\phi)}
\end{equation}

The action must preserv the invariance under diffemorphism, which is why the symmetric part of the connection goes indirectly throught the 
covariant derivative. The fundamental fields to build the action are $\nabla, \mathcal{A}, \mathcal{B}, \mathrm{d}V$. Then we perform a sort
of \textit{dimensional structural analysis technique} studying everysingle possible non-trivial contribution to the action.

THen, the most general action in $3+1$ dimension up to boundary terms is
\begin{equation*}
    \begin{split}
      S
      & =
      \int  \mathrm{d}V^{\alpha \beta \gamma \delta} \bigg[
      B_1 \mathcal{R}_{\mu\nu}{}^{\mu}{}_{\rho}\mathcal{B}_{\alpha}{}^{\nu}{}_{\beta}\mathcal{B}_{\gamma}{}^{\rho}{}_{\delta}
      + B_2 \mathcal{R}_{\alpha\beta}{}^{\mu}{}_{\rho} \mathcal{B}_{\gamma}{}^{\nu}{}_{\delta} \mathcal{B}_{\mu}{}^{\rho}{}_{\nu}
      + B_3 \mathcal{R}_{\mu\nu}{}^{\mu}{}_{\alpha} \mathcal{B}_{\beta}{}^{\nu}{}_{\gamma} \mathcal{A}_\delta
      + B_4 \mathcal{R}_{\alpha\beta}{}^{\sigma}{}_{\rho}\mathcal{B}_{\gamma}{}^{\rho}{}_{\delta}\mathcal{A}_\sigma
      \\
      & \quad
      + B_5 \mathcal{R}_{\alpha \beta}{}^{\rho}{}_{\rho} \mathcal{B}_{\gamma}{}^{\sigma}{}_{\delta} \mathcal{A}_\sigma
      + C_1 \mathcal{R}_{\mu\alpha}{}^{\mu}{}_{\nu} \nabla_\beta \mathcal{B}_{\gamma}{}^{\nu}{}_{\delta}
      + C_2 \mathcal{R}_{\alpha\beta}{}^{\rho}{}_{\rho} \nabla_\sigma \mathcal{B}_{\gamma}{}^{\sigma}{}_{\delta}
      + D_1 \mathcal{B}_{\nu}{}^{\mu}{}_{\lambda} \mathcal{B}_{\mu}{}^{\nu}{}_{\alpha} \nabla_\beta \mathcal{R}_{\gamma}{}^{\lambda}{}_{\delta}
      \\
      & \quad
      + D_2 \mathcal{B}_{\alpha}{}^{\mu}{}_{\beta} \mathcal{B}_{\mu}{}^{\lambda}{}_{\nu} \nabla_{\lambda} \mathcal{B}_{\gamma}{}^{\nu}{}_{\delta}
      + D_3 \mathcal{B}_{\alpha}{}^{\mu}{}_{\nu}\mathcal{B}_{\beta}{}^{\lambda}{}_{\gamma} \nabla_\lambda \mathcal{B}_{\mu}{}^{\nu}{}_{\delta}
      + D_4 \mathcal{B}_{\alpha}{}^{\lambda}{}_{\beta}\mathcal{B}_{\gamma}{}^{\sigma}{}_{\delta}\nabla_\lambda \mathcal{A}_\sigma
      + D_5 \mathcal{B}_{\alpha}{}^{\lambda}{}_{\beta} \mathcal{A}_\sigma \nabla_\lambda \mathcal{B}_{\gamma}{}^{\sigma}{}_{\delta}
      \\
      & \quad
      + D_6 \mathcal{B}_{\alpha}{}^{\lambda}{}_{\beta}\mathcal{A}_\gamma \nabla_\lambda A_\delta
      + D_7\mathcal{B}_{\alpha}{}^{\lambda}{}_{\beta} \mathcal{A}_\lambda \nabla_\gamma A_\delta
      + E_1\nabla_\rho \mathcal{B}_{\alpha}{}^{\rho}{}_{\beta} \nabla_\sigma \mathcal{B}_{\gamma}{}^{\sigma}{}_{\delta}
      + E_2 \nabla_\rho \mathcal{B}_{\alpha}{}^{\rho}{}_{\beta} \nabla_\gamma \mathcal{A}_\delta
      \\
      &\quad
      + F_1 \mathcal{B}_{\alpha}{}^{\mu}{}_{\beta} \mathcal{B}_{\gamma}{}^{\sigma}{}_{\delta} \mathcal{B}_{\mu}{}^{\lambda}{}_{\rho} \mathcal{B}_{\sigma}{}^{\rho}{}_{\lambda}
      + F_2\mathcal{B}_{\alpha}{}^{\mu}{}_{\beta} \mathcal{B}_{\gamma}{}^{\nu}{}_{\lambda} \mathcal{B}_{\delta}{}^{\lambda}{}_{\rho} \mathcal{B}_{\mu}{}^{\rho}{}_{\nu}
      + F_3 \mathcal{B}_{\nu}{}^{\mu}{}_{\lambda} \mathcal{B}_{\mu}{}^{\nu}{}_{\alpha} \mathcal{B}_{\beta}{}^{\lambda}{}_{\gamma} \mathcal{A}_\delta
      + F_4 \mathcal{B}_{\alpha}{}^{\mu}{}_{\beta}\mathcal{B}_{\gamma}{}^{\nu}{}_{\delta}\mathcal{A}_\mu \mathcal{A}_\nu \bigg].
    \end{split}
  \end{equation*}

In previous works we have mentioned some of the features of the action: (i) Its rigidity, since contains all possible combinations of the fields and their derivatives; 
(ii) All the coupling constants are dimensionless, which might be a sign of conformal invariance, and also ensure that the model is power-counting renormalisable; 
(iii) The field equations are second order differential equations, and the Einstein spaces are a subset of their solu- tions; 
(iv) The supporting symmetry group is the group of diffeomorphisms, desirable for the background independence of the model; 
(v) Even though there is no fundamental metric, it is possible to obtain emergent (connection- descendent) metric tensors; 
(vi) The cosmological constant appears in the solutions as an integration constant, changing the paradigm concerning its interpretation; 
(vii) The model can be extended to be coupled with a scalar field, and the field equations are equivalent to those of General Relativity interacting with a massless scalar field. 



To couple the affine action to a scalar field, we need to introduce a kinetic term in the absence of the metric tensor. In order to do so, we build 
\textit{inverse symmetric tensor densities}, by using the \textit{dimensional analysis structure technique}

\begin{equation}
      \mathrm{g}^{\mu\nu} = \left(\alpha \nabla_\lambda \mathcal{B}_{\rho}{ }^{\mu}{ }_{\sigma} + \beta \mathcal{A}_\lambda 
      \mathcal{B}_{\rho}{ }^{\mu}{ }_{\sigma}\right)\mathrm{d}V^{\nu\lambda\rho\sigma} + \gamma \mathcal{B}_{\kappa}{ }^{\mu}{ }_{\lambda}
      \mathcal{B}_{\rho}{ }^{\nu}{ }_{\sigma}\mathrm{d}V^{\kappa\lambda\rho\sigma}
\end{equation}

Using the above expression we can define the kinetic term 

\begin{equation}
    S_{\phi} = - \int \mathrm{g}^{\mu\nu} \partial_{\mu} \phi \partial_{\nu} \phi
\end{equation}

Since we want to work on the torsion-free sector, it is worth to notice that only the terms that are linear in the torsion will
have a non-trivial contribution, $C_1$ and $C_2$. Additionally, since our connection its an \textit{equi-affine} connection, the trace of the
Riemman tensor will vanish completly. Applying the same idea the to the scalar field action, only the $\alpha$ term survive. Thus,
the effective action coupled with a scalar field is

\begin{equation*}
    \begin{split}
      S_{ef}
      & =
      \int  \mathrm{d}V^{\alpha \beta \gamma \delta} \bigg[
      C_1 \mathcal{R}_{\mu\alpha}{}^{\mu}{}_{\nu} - \alpha \partial_{\alpha}\phi \partial_{\nu}
      \bigg]\nabla_\beta \mathcal{B}_{\gamma}{}^{\nu}{}_{\delta}.
    \end{split}
  \end{equation*}


\subsection{Cosmological ansatz}

In order to solve the field equations, one need to build an ansatz, since we want to do cosmology, we need to build an ansazts
compatible with the symmetries of the cosmological principle, which are rotation and translations. It is possible to build an ansatz for our
fundamental geometric objects using the Lie derivative along the Killing vector fields. The most general ansatz for the symmetric part of the 
connection $\Gamma_{\mu}{}^{\sigma}{}_{\nu}$ is
\begin{align}
      \Gamma_{t}{}^{t}{ }_{t} & =f(t), \quad \Gamma_{i}{ }^{t}{ }_{j}=g(t) S_{i j} \\
      \Gamma_{t}{ }^{i}{ }_{j} &= h(t) \delta^{i}_{j}, \quad \Gamma_{i}{ }^{j}{ }_{k}= \gamma_{i}{ }^{j}{ }_{k}
\end{align}
Since we are going to be working in the torsio-free sector, in principle there is no neccesity to compute the ansatz for the torsion tensor, however, just
for the sake of completness, we are going to include them. The traceless part of the torsion tensor $\mathcal{B}_{\mu}{}^{\sigma}{}_{\nu}$ is completely define by only one time depending function
\begin{align*}
    \mathcal{B}_{\theta}{ }^{r}{ }_{\varphi} & = \psi (t) r^2\sin\theta \sqrt{1 - \kappa r^2} &
    \mathcal{B}_{r}{}^{\theta}{}_{\varphi} & =\frac{\psi (t) \sin \theta}{\sqrt{1 - \kappa r^2}} \\
    \mathcal{B}_{r}{}^{\varphi}{}_{\theta} & =\frac{\psi(t)}{ \sqrt{1-\kappa r^{2}} \sin \theta}
\end{align*}
and finally, the vectorial torsion tensor $\mathcal{A}_\mu$ is given by
\begin{align}
    \mathcal{A}_{t} = \eta(t)
\end{align}

\subsection{The field equations}

The field equations are obtained using Kiwosjki's formalism and taking into account the symmetries and properties of the fundamental fields, we 
vary the action with respect to the fundamental fields. In the torsion-free limit, the field equation is
\begin{equation}
  \nabla_\mu \biggl[\frac{1}{\mathcal{V}(\phi)} \left(C \partial_\alpha \phi \partial_\lambda \phi - \mathcal{R}_{\alpha\lambda}\right)\mathrm{d}V^{\mu\nu\rho\alpha}\biggr] 
  + \frac{2}{3}\nabla_\mu \biggl[ \frac{1}{\mathcal{V}(\phi)}\mathcal{R}_{\alpha\theta} \delta^{[\nu}_{\lambda}\mathrm{d}V^{\rho]\alpha\mu\theta} \biggr] = 0
\end{equation}
By multiplying the left hand side of the field equation by $\epsilon_{\nu\rho\tau\beta}$, the second term vanishes completly and the field 
equation is reduced even further to
\begin{equation}
  \nabla_{[\mu}\left(\mathcal{R}_{\nu]\gamma}\frac{1}{\mathcal{V}(\phi)}\right) 
  - C \nabla_{[\mu}\left(\partial_{\nu]} \phi \partial_\gamma \phi \frac{1}{\mathcal{V}(\phi)}\right) = 0
\end{equation}
This is the field equation in Polynomial Affine Graivity coupled with a scalar field in the torsion-free sector.




\section{Cosmological Solutions}

The field equation in the torsion-free sector coupled with a scalar field is given by
\begin{equation}
  \nabla_{[\mu}\left(\mathcal{R}_{\nu]\gamma}\frac{1}{\mathcal{V}(\phi)}\right) 
  - C \nabla_{[\mu}\left(\partial_{\nu]} \phi \partial_\gamma \phi \frac{1}{\mathcal{V}(\phi)}\right) = 0
\end{equation}
which can be rewritten as
\begin{equation}
  \nabla_{[\mu}\mathcal{S}_{\nu]\gamma} = 0
\end{equation}
where the $\mathcal{S}_{\mu\nu}$ tensor is a symmetric tensor defined by
\begin{equation}
  \mathcal{S}_{\mu\nu} = \left(\mathcal{R}_{\mu\nu} - C \partial_{\mu} \phi \partial_\nu \phi \right)\frac{1}{\mathcal{V}(\phi)}
\end{equation}
From here, we distinguish three types of solutions
\begin{align}
  \mathcal{S}_{\mu\nu} & = 0 & \nabla_{\eta} \mathcal{S}_{\mu\nu} & = 0 & \nabla_{[\eta} \mathcal{S}_{\mu]\nu} & = 0
\end{align}

Notice that up to this point we have not use any metric tensor in our formulation and field equations, however, we still have an emergent metric 
tensor coming from the Ricci tensor. When the Ricci tensor $\mathcal{R}_{\mu\nu}$ its symmetric and no degenerate, it serves as a metric tensor.
\begin{align}
  \mathcal{R}_{tt} & = -3\left(\dot{h} + h^2\right) & \mathcal{R}_{rr} & = \frac{\dot{g} + gh + 2\kappa}{1 - \kappa r^2}
\end{align}
Notice that the expression $\dot{g} + gh + 2\kappa$ its the affine analogue to the well-known scale factor $a(t)$ in the FRW universe. As 
one does in \textit{General Relativity}, demanding that its covariant derivative must vanishes completly, we are able 
to completely define the affine functions of the symmetric part of the connection
\begin{align}
  h(t) & = \sqrt{A_1}\tanh\left(t\sqrt{A_1}\right) & g(t) & = \frac{\kappa \sinh\left(2t\sqrt{A_1}\right) }{2\sqrt{A_1}}
\end{align}
The above definitions ensures that $\nabla_\alpha \mathcal{R}_{\beta\gamma} = 0$, this asservs the metricity condition. Then, the Ricci tensor
is written as
\begin{align}
  \mathcal{R}_{tt} & = -3A_1 & \mathcal{R}_{rr} & =  \frac{3\kappa \cosh^2(t\sqrt{A_1})}{1 - \kappa r^2}
\end{align}
If we require that the Ricci tensor acts as a metric tensor and we demand that its covariantly constant, then 
the only unknown functions are the scalar field $\phi$ and its potential energy term $\mathcal{V}(\phi)$.

\subsection{$\mathcal{S}_{\mu\nu} = 0 $}

In the case where the symmetric tensor $\mathcal{S}_{\mu\nu}$ vanishes completely we are left with two differential equations
\begin{align}
  \dot{h} + h^2 + C\dot{\phi}^2 & = 0 & \dot{g} + gh + 2\kappa & = 0
\end{align}
where the potential energy $\mathcal{V}(\phi)$ does not play any role. Here we have two unknown functions of time and the scalar field, the best we 
can do is to solve the scalar field $\phi(t)$ in terms of the time derivative of the $h(t)$ function , as well as the $g(t)$ in terms of $h(t)$
\begin{align}
  \phi(t) & = \phi_0 \pm \int \mathrm{d}t \sqrt{\frac{\dot{h} + h^2}{-C}} & 
  g(t) & = e^{- \int \mathrm{d}t \, h} \left( g_0 - 2 \kappa \int \mathrm{d}t \, e^{\int \mathrm{d}t \, h} \right)
\end{align}
However, if we impose that the Ricci tensor is covariantly constant, the above equation is reduced to
\begin{align}
  C\dot{\phi}^2(t) + A_1 & = 0 & 3\kappa \cosh(t\sqrt{A_1}) & = 0
\end{align}
from the second equation its clear to see that $\kappa = 0$, and from the first equation we can determined the scalar field
\begin{equation}
  \phi(t) = \pm\sqrt{-\frac{A_1}{C}}t + \phi_0
\end{equation}


\subsection{$\nabla_{\eta} \mathcal{S}_{\mu\nu} = 0 $}

The second case its the affine parallel scenario. Here we have three equations
\begin{align}
  0 & = C\mathcal{V}'\dot{\phi}^3-2C\mathcal{V}\dot{\phi}\ddot{\phi} - 3\mathcal{V}\left(2h\dot{h} + \ddot{h}\right) + 3\mathcal{V}'\dot{\phi}
  \left(h^{2} + \dot{h}\right) \\
  0 & = Cg\dot{\phi}^2 + 2gh^2 - 2\kappa h - h\dot{g} + 3g\dot{h} \\
  0 & = \left(2gh^2 + 4\kappa h +h\dot{g} - g \dot{h} - \ddot{g}\right)\mathcal{V} + \left(gh + 2\kappa + \dot{g}\right)\mathcal{V}'\dot{\phi}
\end{align}
Demanding that the Ricci tensor covariant derivative vanishes, the equations reduced to
\begin{align}
  0 & = C\mathcal{V}'\dot{\phi}^3 - 2C\mathcal{V}\dot{\phi}\ddot{\phi} + 3A_1 \mathcal{V}'\dot{\phi}\\
  0 & = C\kappa \cosh(t\sqrt{A_1})\sinh(t\sqrt{A_1})\dot{\phi} \\
  0 & = \kappa \cosh(t\sqrt{A_1})\mathcal{V}'\dot{\phi}
\end{align}
the last two equations can be solve simultaneously by fixing $\kappa = 0$, additionally, the first equation can be rewritten as
\begin{equation}
  \ddot{\phi} - \frac{1}{2}\dot{\phi}^2\left(\frac{\mathcal{V}'}{\mathcal{V}}\right) + \frac{3A_1}{2C}\left(\frac{\mathcal{V}'}{\mathcal{V}}\right) = 0
\end{equation}
Here, just like before we can give the potential energy term $\mathcal{V}(\phi)$, taking the potential \textit{Power-Law} the scalar field
can be solved in terms of inverse hypergeometric functions, but for $n = 0 , 1 , 2$ we have an analytically expression. In the case of Starobinsky
potential, the scalar field is written in terms of inverse hyperbolic tangent function. Notice that in this situation, we need to constraint $\kappa = 0$ to have non-trivial field equation, this, in some sense matches what
happens with the $\kappa$ factor in FRW coupled with a scalarfield. 

One could go even further and try to use the \textit{slow-roll} conditions, meaning that the kinetic energy of the scalar field its 
neglectable with respect to the potential energy
\begin{align}
  \mathcal{V}(\phi) & \gg  \frac{1}{2}\dot{\phi}^2 & \mathcal{V}'(\phi) & \gg \ddot{\phi}
\end{align}
the second constraint was obtained from the first condition. Applying the above approximation to the field equations
\begin{equation}
  \left(\frac{\mathcal{V}'}{\mathcal{V}}\right)\left(\frac{3A_1}{2C\dot{\phi}^2} -1 \right)= 0
\end{equation}
from here we have to type of solutions, the first one requires that the potential energy term $\mathcal{V}(\phi)$ must be a constant. The second
type of solution requires that the scalar field its a linear function of time (just like in the affine reduced case), in this context, the scalar field its explicitly determined
as a function of time, whereas the potential energy term remains undetermined. 

We could try to solve the parallel equations by considering another well-known approximation, in FRW coupled with a scalar field, in order to ensure
to have an exponential growth of the scale factor, the hubble parameter must be a constant. In our affine formulation, this approximation its translate to demanding
that the affine function $h(t) = h_0$ must be a constant. Then given a potential $\mathcal{V}(\phi)$ like a power-law type, the system can be solved analitically. Additionally, the covariant 
derivative of the Ricci tensor does not vanishes completely, however, the spatial part of the Ricci tensor vanishes, meaning that its degenerate. 

Another way to solve the equations, is by taking the slow-roll approximation first and then give the potential energy.
\begin{align}
  0 & = C\mathcal{V}'\dot{\phi}^3-2C\mathcal{V}\dot{\phi}\ddot{\phi} - 3\mathcal{V}\left(2h\dot{h} + \ddot{h}\right) + 3\mathcal{V}'\dot{\phi}
  \left(h^{2} + \dot{h}\right) \\
  0 & = Cg\dot{\phi}^2 + 2gh^2 - 2\kappa h - h\dot{g} + 3g\dot{h} \\
  0 & = \left(2gh^2 + 4\kappa h +h\dot{g} - g \dot{h} - \ddot{g}\right)\mathcal{V} + \left(gh + 2\kappa + \dot{g}\right)\mathcal{V}'\dot{\phi}
\end{align}
Using the slow-roll approximation and working for the case where the potential energy term is given by a \textit{Power-Law} in a flat space-time the field equations are 
reduced to
\begin{align}
  0 & = -2h\dot{h} - \ddot{h} + \frac{\dot{\phi}}{\phi}n\left(h^2 + \dot{h}\right) \\
  0 & = g\left(2h^2 + 3\dot{h} + C\dot{\phi}^2\right) - h\dot{g}\\
  0 & = h\left(2gh + \dot{g}\right) - g\dot{h} + \frac{\dot{\phi}}{\phi}n\left(gh + \dot{g}\right) - \ddot{g}
\end{align}


\subsection{$\nabla_{[\eta} \mathcal{S}_{\mu]\nu} = 0 $}

Under the symmetries of the cosmological ansatz we only have one equation 
\begin{equation}
  C\mathcal{V}(\phi)g\dot{\phi}^2 + \mathcal{V}(\phi)\left(4gh^2 +2\kappa h + 2g\dot{h} - \ddot{g}\right) +
  \mathcal{V}'(\phi)\dot{\phi}\left(gh + 2\kappa + \dot{g}\right) = 0
\end{equation}
Since we have two unknown functions of time and the scalar field, it is not possible to solve the above equation. However, if the Ricci tensor
it is not degenerate, then it can serve the function as a metric tensor. Therefore, demanding that is covariant derivative must be zero, 
the above equation is reduced to
\begin{equation} 
  \dot{\phi}(t) + \frac{3\sqrt{A_1}}{C}\tanh\left(t\sqrt{A_1}\right)\left(\frac{\mathcal{V}'(\phi)}{\mathcal{V}(\phi)}\right) = 0
\end{equation}
At this point we can proceed as one usually does in classical cosmology, given a potential you find the scale factor and the scalar field, whereas 
here given a potential, you only need to determined the scalar field. The most well known potentials are the \textit{Power-Law potential}
\textit{Starobinsky potential}, and the \textit{Intermediate Inflation}

Taking first the \textit{Power-Law potential} meaning $\mathcal{V}(\phi) = \beta \phi^n(t)$, then the above equation leads as
\begin{equation}
  \dot{\phi}(t) + \frac{3\sqrt{A_1}}{C}\tanh\left(t\sqrt{A_1}\right)\left(\phi^{-1}(t)n\right) = 0
\end{equation}
which can be solved analytically
\begin{equation}
  \phi(t) = \pm \sqrt{\frac{2C\phi_0 - 6n\log(\cosh(t\sqrt{A_1}))}{C} }
\end{equation}

Then we take \textit{Starobinsky potential} written as $\mathcal{V}(\phi) =  \alpha \left(1 - e^{-\beta\phi}\right)^2$, the equation takes
the form of
\begin{equation}
  \dot{\phi}(t) - \frac{6\beta\sqrt{A_1}}{C}\tanh\left(t\sqrt{A_1}\right)\left(\frac{1}{\left(1 - e^{-\beta\phi}\right)}\right) = 0
\end{equation}
which can be solved analytically in terms of the inverse hypergeometric functions.

Then we take \textit{Intermediate inflation} written as $\mathcal{V}(\phi) =  \alpha \phi^{4\left(1-\frac{1}{f}\right)}$, the equation takes
the form of
\begin{equation}
  \dot{\phi}(t) - \frac{6\beta\sqrt{A_1}}{C}\tanh\left(t\sqrt{A_1}\right)\left(\frac{\alpha \left(4\left(1 - \frac{1}{f}\right)\right) e^{-1 + 4\left(1-\frac{1}{f}\right)}}{\alpha e^{4\left(1-\frac{1}{f}\right)}}\right) = 0
\end{equation}
which can be solved analytically in 
\begin{equation}
  \phi(t) = \pm \sqrt{\frac{2fC\phi_0 + 24(f-1)\ln(\cosh(t\sqrt{A_1}))}{fC}}
\end{equation}

Notice that for the above well-known potential in the cosmology literature, in our formulation it is a condition that $\kappa \neq 0$ in order
to have non-trivial field equation, whereas in FRW coupled with a scalar field, its a constraint that $\kappa = 0$, these is a strong difference
between our geometries. Additionally, while in FRW one needs to impose the \textit{slow-roll} conditions to solve the field equations, here
in the absence of those approximation we are able to solve the field equations.

\section{Conclusions}


\section{Appendix A: Dimensional Analysis}

In this section we briefly show how to build the action and coupling the scalar field in the absence of the metric tensor, by using a sort of
\textit{dimensional analysis} technique.

\subsection{Building the action}

In order to build the most general action while preserving the invariance under diffeormophism, we perform a \textit{dimensional analysis} technique. First,
we define an operator $\mathcal{N}$ to count the number of free index  and a second operator $\mathcal{W}$ to define the weight density of the object. Applying both
operators to the fundamental fields leads to
\begin{align}
  \mathcal{N}(\mathcal{A}_\mu)& = -1 & \mathcal{N}(\mathcal{B}_{\mu}{}^{\lambda}{}_{\nu})& = -1 & \mathcal{N}(\Gamma_{\mu}{}^{\lambda}{}_{\nu})& = -1 & \mathrm{d}V^{\alpha\beta\gamma\delta}& = 4 \\
  \mathcal{W}(\mathcal{A}_\mu)& = 0 & \mathcal{W}(\mathcal{B}_{\mu}{}^{\lambda}{}_{\nu})& = 0 & \mathcal{W}(\Gamma_{\mu}{}^{\lambda}{}_{\nu})& = 0 & \mathrm{d}V^{\alpha\beta\gamma\delta}& = 1 
\end{align}
A generic term will have the following form
\begin{equation}
  \mathcal{O} = \mathcal{A}^m\mathcal{B}^n\Gamma^p\mathrm{d}V^q
\end{equation}
Applying the operators defined above, yield the equations
\begin{align}
  \mathcal{N}(\mathcal{O}) & = 4q - m - n - p & \mathcal{W}(\mathcal{O}) & = q 
\end{align}
Notice that we are interested in building scalar densities, meaning that the number of free index must be zero and the weight density must be equal to the unity. Therefore,
we have two constraints
\begin{align}
  m + n + p & = 4 & q & = 1
\end{align}
The terms contributing to the action are shown in the below table
\begin{table}[h]
\begin{center}
  \begin{tabular}{ | p{1cm} | p{1cm} | p{1cm} | p{3.5cm} | p{2cm} |}
  \hline
  $\mathcal{A}^m$ & $\mathcal{B}^n$ & $\Gamma^p$ & Type of configuration & Action term\\ \hline
  4 & 0 & 0 & $\mathcal{A}\mathcal{A}\mathcal{A}\mathcal{A}$ & 0 \\
  3 & 1 & 0 & $\mathcal{A}\mathcal{A}\mathcal{A}\mathcal{B}$ & 0 \\
  3 & 0 & 1 & $\mathcal{A}\mathcal{A}\mathcal{A}\nabla$ & 0  \\
  2 & 2 & 0 & $\mathcal{A}\mathcal{A}\mathcal{B}\mathcal{B}$ & $F_4$ \\
  2 & 1 & 1 & $\mathcal{A}\mathcal{A}\mathcal{B}\nabla$ & $D_6,D_7$ \\
  2 & 0 & 2 & $\mathcal{A}\mathcal{A}\nabla\nabla $ & 0    \\
  1 & 3 & 0 & $\mathcal{A}\mathcal{B}\mathcal{B}\mathcal{B}$ & $F_3$ \\
  1 & 2 & 1 & $\mathcal{A}\mathcal{B}\mathcal{B}\nabla$ & $D_4,D_5$ \\
  1 & 1 & 2 & $\mathcal{A}\mathcal{B}\nabla\nabla$ & $ B_3,B_4,B_5,E_2$\\
  1 & 0 & 3 & $\mathcal{A}\nabla\nabla\nabla$ & 0  \\
  0 & 4 & 0 & $\mathcal{B}\mathcal{B}\mathcal{B}\mathcal{B}$ & $F_1,F_2$ \\
  0 & 3 & 1 & $\mathcal{B}\mathcal{B}\mathcal{B}\nabla$ & $D_1,D_2,D_3$\\
  0 & 2 & 2 & $\mathcal{B}\mathcal{B}\nabla\nabla$ & $B_1,B_2,E_1$ \\
  0 & 1 & 3 & $\mathcal{B}\nabla\nabla\nabla$ & $C_1,C_2$ \\
  0 & 0 & 4 & $\nabla\nabla\nabla\nabla$ & 0\\ \hline
  \end{tabular}
  \caption{Possible terms contributing to the action of Polynomial Affine gravity}
\end{center}
\end{table}

From the above table, one use the symmetries of the tensor to see which terms will have non trivial contribution to the action. For example, the term with
four $\mathcal{A}$ does not contribute to the action since its contraction with the volume element is identically zero. Whenever two covariant
derivatives are contracted with the volume form they give a curvature tensor, and since the curvature is defined for the symmetric part of the connection, such 
curvature satisfy the torsion-free Bianchi identities, which relate some of the several possible contractions of the indices. An  additional argument that 
helps to drop contraction of indices is that $\mathcal{B}$ is traceless.

\subsection{Coupling the scalar field $\phi$}

In order to form the kinetic term without the metric tensor we proceed just like the subsection before, however, here we are interested in building
tensor densities of the type $(0,2)$, and additionally we require that it has to be symmetric. Then, the constraint equations are
\begin{align}
  m + n + p & = 2 & q & = 1
\end{align}
From here we present the table with all possible configuration
\begin{table}[h]
  \begin{center}
    \begin{tabular}{ | p{1cm} | p{1cm} | p{1cm} | p{3.5cm} |}
    \hline
    $\mathcal{A}^m$ & $\mathcal{B}^n$ & $\Gamma^p$ & Type of configuration \\ \hline
    2 & 0 & 0 & $\mathcal{A}\mathcal{A}$  \\
    1 & 1 & 0 & $\mathcal{A}\mathcal{B}$  \\
    1 & 0 & 1 & $\mathcal{A}\nabla$    \\
    0 & 2 & 0 & $\mathcal{B}\mathcal{B}$  \\
    0 & 1 & 1 & $\mathcal{B}\nabla$   \\
    0 & 0 & 2 & $\nabla\nabla $     \\ \hline
    \end{tabular}
    \caption{Possible terms to contract the kinetic term of the scalar field $\phi$}
  \end{center}
  \end{table}
The first type of term, can only be configurate in one form where two vectorial torsion $\mathcal{A}$ must be contracted with the volumen element
\begin{equation*}
    \mathcal{A}_\alpha\mathcal{A}_\beta\mathrm{d}V^{\alpha\beta\mu\nu} 
\end{equation*}
meaning that its contributio vanishes completely. The second type, produces only two possible terms
\begin{equation*}
    \mathcal{A}_\sigma\mathcal{B}_{\alpha}{}^{\sigma}{}_{\beta}\mathrm{d}V^{\alpha\beta\mu\nu} + 
    \mathcal{A}_\alpha\mathcal{B}_{\beta}{}^{\mu}{}_{\gamma}\mathrm{d}V^{\alpha\beta\gamma\nu}
\end{equation*} 
The first term can not be used because it is an antisymmetric object, therefore its coupling with the kinetic term of the scalar field
wil vanishes completely, whereas the second produce non trivial contribution. The thir type term has only one contribution
\begin{equation*}
  \nabla_\alpha \mathcal{A}_\beta\mathrm{d}V^{\alpha\beta\mu\nu}
\end{equation*}
which is trivial. The fourth type has two possible configuration
\begin{equation*}
  \mathcal{B}_{\alpha}{}^{\mu}{}_{\beta}\mathcal{B}_{\gamma}{}^{\nu}{}_{\delta}\mathrm{d}V^{\alpha\beta\gamma\delta} + 
  \mathcal{B}_{\alpha}{}^{\sigma}{}_{\beta}\mathcal{B}_{\sigma}{}^{\mu}{}_{\gamma}\mathrm{d}V^{\alpha\beta\gamma\nu} 
\end{equation*}
Here both term have non trivial contribution, however, since we are going to be dealing with the torsion-free sector, both term will no have
an effective action on the field equaitons. The fith configuration has two possible terms
\begin{equation*}
  \nabla_\alpha \mathcal{B}_{\beta}{}^{\mu}{}_{\gamma}\mathrm{d}V^{\alpha\beta\gamma\nu} + 
  \nabla_\sigma \mathcal{B}_{\alpha}{}^{\sigma}{}_{\beta}\mathrm{d}V^{\alpha\beta\mu\nu}
\end{equation*}
Notice that second term will produce a trivial term since its antisymmetric in its two free indices, whereas the first term have a non trivial contribution. Finally,
the las type
\begin{equation*}
  \mathcal{R}_{\alpha\beta}{}^{\mu}{}_{\gamma}\mathrm{d}V^{\alpha\beta\gamma\nu} + 
  \mathcal{R}_{\alpha\beta}{}^{\sigma}{}_{\sigma}\mathrm{d}V^{\alpha\beta\mu\nu}
\end{equation*}
both terms are trivial through Bianchi's identity and the antisymmetry of the volumen element. 

\section{Appendix B: GR as a subspace of PAG}

The field equation is given by
\begin{equation}
  \nabla_{[\mu}\left(\mathcal{R}_{\nu]\gamma}\frac{1}{\mathcal{V}(\phi)}\right) 
  - C \nabla_{[\mu}\left(\partial_{\nu]} \phi \partial_\gamma \phi \frac{1}{\mathcal{V}(\phi)}\right) = 0
\end{equation}
A particular solution of the above field equation is given by
\begin{equation}
  \nabla_{\mu}\left(\mathcal{R}_{\nu\gamma}\frac{1}{\mathcal{V}(\phi)}\right) 
  - C \nabla_{\mu}\left(\partial_{\nu} \phi \partial_\gamma \phi \frac{1}{\mathcal{V}(\phi)}\right) = \nabla_\mu \left(\Lambda g_{\nu\gamma}\right)
\end{equation}
Notice that in order to be consistent, the object $g_{\mu\nu}$ must be symmetric, so taking the antysymmetrize covariant derivative it vanishes completely, 
allowing us to recover the field equation. This gives rise a metric tensor named $g_{\mu\nu}$. Next, a particular solution of the above equation is
\begin{equation}
  \mathcal{R}_{\mu\nu} - C\partial_\mu \phi \partial_\nu \phi = \Lambda \mathcal{V}(\phi) g_{\mu\nu}
\end{equation}
and by using the metric tensor, the above equation can be rewritten in a more well-known equation 
\begin{equation}
  \mathcal{R}_{\mu\nu} - \frac{1}{2}\mathcal{R}g_{\mu\nu} + \Lambda \mathcal{V}(\phi) g_{\mu\nu} = 
  C\left(\partial_\mu \phi \partial_\nu \phi - \frac{1}{2}g_{\mu\nu} (\partial \phi)^2\right)
\end{equation}

If we vary the action with respect to the scalar field and take the torsion-free limit, then we will have a trivial contribution, however, we can still
have the field equation for the scalar field by proceeding in a different manner. Taking the divergence $\nabla^\mu$ of the above equation leads to
\begin{equation}
  \Lambda g_{\mu\nu}\nabla^{\mu}\mathcal{V}(\phi) =C \left(\nabla^{\mu}\partial_\mu \phi \partial_\nu \phi + \partial_\mu \phi \nabla^{\mu}\partial_\nu \phi
  - g_{\mu\nu} \partial \phi \nabla^{\mu}\partial \phi \right)
\end{equation}
after some simplification, the above equation leads to
\begin{equation}
  C \nabla^\mu \nabla_\mu \phi - \Lambda \mathcal{V}'(\phi) = 0
\end{equation}
which is the Klein-Gordon field equation. Notice that in order to obtain the above equation we require the existence of the $g_{\mu\nu}$ object, to define the 
d'Alembert operator. Without this object, it is not possible to obtain a Klein-Gordon field equation, additionally we require that the integration 
constant $\Lambda \neq 0$ and also we demand that the covariant derivative of $g_{\mu\nu}$ vanishes
completely, to satisfy the metricity condition.


Since, we have the tensor $g_{\mu\nu}$ presented as a particular solution to the field equation. Thus, we need to provied
an ansatz for this tensor compatible with the symmetries of the cosmological principle. This is donde by computing the Lie derivative of an 
arbitrary tensor type $(0,2)$ along the Killing vectors that generate the symmetries of rotation and translations. This compute leads to 
\begin{equation}
  g_{\mu\nu} = b(t)\mathrm{d}t^2 + a(t)\left(\frac{1}{1 - \kappa r^2} \mathrm{d}r^2 + r^2 \mathrm{d}\theta^2 
  + r^2 \sin^2 \theta \mathrm{d} \varphi^2 \right)
\end{equation}
Demanding that its covariant derivative vaniahes, we found restriction for the above functions in terms of the connection coefficients
\begin{align}
  b(t) & = -b_0 & h(t) & = \frac{\dot{a}(t)}{a(t)} & g(t) & = \frac{\dot{a}(t)a(t)}{b_0}
\end{align}

Replacing the cosmological ansatz for $g$ and $\Gamma$ in the field equation yields to
\begin{align}
  0 & = 3\ddot{a}(t) + C a(t)\dot{\phi}^{2}(t) - \Lambda b_0 a(t)\mathcal{V}(\phi) \\
  0 & = \ddot{a}(t)a(t) + 2\dot{a}^2(t) + 2b_0\kappa - \Lambda b_0 a^2(t)\mathcal{V}(\phi)
  \end{align}
Combining the two equations we obtained we get
\begin{equation}
  3H^2(t) = \Lambda b_0 \mathcal{V}(\phi) -\frac{1}{2}C \dot{\phi}^2(t) - \frac{3b_0\kappa}{a^2(t)}
\end{equation}
where the first function is the Hubble function $H(t)$, here we are able to recover the FRW field equation coupled with a scalar field. Additionally,
the field equation for the scalar is given by the Klein-Gordon field equation, replacing the cosmological ansatz leads to
\begin{equation}
  \ddot{\phi}(t) + 3H(t)\dot{\phi}(t) + \Lambda b_0 \mathcal{V}'(\phi) = 0
\end{equation}
Thus we are able to recover Einstein-Hillbert coupled with a scalar field.



\end{document}