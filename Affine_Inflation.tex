\documentclass[10pt,a4paper]{article}
\usepackage[utf8]{inputenc}
\usepackage{amsmath}
\usepackage{amsfonts}
\usepackage{amssymb}

\usepackage{vmargin}

\usepackage{graphicx}
\graphicspath{ {./images/} }


\usepackage{slashed}


\title{Affine Inflation in Polynomial Affine Gravity in $3+1$ dimensions}
\author{Jose Perdiguero Garate}
\date{20/12/2022}			

\begin{document}

\maketitle

\begin{abstract}
  The Polynomial Affine Gravity its a purely affine model that mediates gravitational interactions solely and exclusive through the
  affine connection instead of the metric tensor. In this paper we couple a scalar field through \textit{inverse tensor densities} and 
  its potential energy to the volume form.
\end{abstract}

\tableofcontents

\section{Introduction}

\section{Polynomial Affine Gravity}

The Polynomial Affine Gravity model its a purely affine model on which we endowed the manifold only with an affine connection 
$(\mathcal{M}, \Gamma)$. This allow us to define the notion of parallelism by the covariant derivative $\nabla$. Since we only
have an affine connection $\Gamma$ we can only deffine the following chain of geometric objects

\begin{equation}
  \Gamma_{\mu}{}^{\sigma}{}_{\nu} \to
  \nabla_\mu \to \mathcal{R}_{\mu\sigma}{}^{\tau}{}_{\nu} \to \mathcal{R}_{\mu\nu}  
\end{equation} 

Notice that in the absence of the metric tensor it is not possible to define the $\mathcal{R}$.

\subsection{The action}

In order to built the action of the Polynomial Affine Gravity we use the irreducible fields of the affine connection,
by separating the connection into its symmetric and antisymmetric part

\begin{equation}
  \hat{\Gamma}_{\mu}{}^{\sigma}{}_{\nu} = \Gamma_{\mu}{}^{\sigma}{}_{\nu} + \mathcal{B}_{\mu}{}^{\sigma}{}_{\nu} + \delta^{\sigma}_{[\mu}\mathcal{A}_{\nu]}
\end{equation}

where $\Gamma_{\mu}{}^{\sigma}{}_{\nu}$ correspond to the symmetric part of the connection, $\mathcal{B}_{\mu}{}^{\sigma}{}_{\nu}$ its the traceless
part of the torsion tensor and $\mathcal{A}_{\mu}$ its the vectorial part of the torsion tensor. Additionally, we need to define the volume
form, which can be written using only the wedge product

\begin{equation}
  \mathrm{d}V^{\alpha\beta\gamma\delta} = J(x)\mathrm{d}x^{\alpha}\wedge\mathrm{d}x^{\beta}\wedge\mathrm{d}x^{\gamma}\wedge\mathrm{d}x^{\delta}
\end{equation}

However, since we want to couple a scalar field $\phi (x)$ to this model, we need to introduce the potential energy of the 
scalar field $\mathcal{V}(\phi)$. Inspired by the work of Hemza Azri in affine inflation, we couple the potential energy to
the volume form in the following manner

\begin{equation}
  \mathrm{d}V^{\alpha\beta\gamma\delta} = \mathrm{d}\hat{V}^{\alpha\beta\gamma\delta}\frac{1}{\mathcal{V}(\phi)}
\end{equation}

The action must preserv the invariance under diffemorphism, which is why the symmetric part of the connection goes indirectly throught the 
covariant derivative. The fundamental fields to build the action are $\nabla, \mathcal{A}, \mathcal{B}, \mathrm{d}V$. Then we perform a sort
of \textit{dimensional structural analysis technique} studying everysingle possible non-trivial contribution to the action.

THen, the most general action in $3+1$ dimension up to boundary terms is
\begin{equation*}
    \begin{split}
      S
      & =
      \int  \mathrm{d}V^{\alpha \beta \gamma \delta} \bigg[
      B_1 \mathcal{R}_{\mu\nu}{}^{\mu}{}_{\rho}\mathcal{B}_{\alpha}{}^{\nu}{}_{\beta}\mathcal{B}_{\gamma}{}^{\rho}{}_{\delta}
      + B_2 \mathcal{R}_{\alpha\beta}{}^{\mu}{}_{\rho} \mathcal{B}_{\gamma}{}^{\nu}{}_{\delta} \mathcal{B}_{\mu}{}^{\rho}{}_{\nu}
      + B_3 \mathcal{R}_{\mu\nu}{}^{\mu}{}_{\alpha} \mathcal{B}_{\beta}{}^{\nu}{}_{\gamma} \mathcal{A}_\delta
      + B_4 \mathcal{R}_{\alpha\beta}{}^{\sigma}{}_{\rho}\mathcal{B}_{\gamma}{}^{\rho}{}_{\delta}\mathcal{A}_\sigma
      \\
      & \quad
      + B_5 \mathcal{R}_{\alpha \beta}{}^{\rho}{}_{\rho} \mathcal{B}_{\gamma}{}^{\sigma}{}_{\delta} \mathcal{A}_\sigma
      + C_1 \mathcal{R}_{\mu\alpha}{}^{\mu}{}_{\nu} \nabla_\beta \mathcal{B}_{\gamma}{}^{\nu}{}_{\delta}
      + C_2 \mathcal{R}_{\alpha\beta}{}^{\rho}{}_{\rho} \nabla_\sigma \mathcal{B}_{\gamma}{}^{\sigma}{}_{\delta}
      + D_1 \mathcal{B}_{\nu}{}^{\mu}{}_{\lambda} \mathcal{B}_{\mu}{}^{\nu}{}_{\alpha} \nabla_\beta \mathcal{R}_{\gamma}{}^{\lambda}{}_{\delta}
      \\
      & \quad
      + D_2 \mathcal{B}_{\alpha}{}^{\mu}{}_{\beta} \mathcal{B}_{\mu}{}^{\lambda}{}_{\nu} \nabla_{\lambda} \mathcal{B}_{\gamma}{}^{\nu}{}_{\delta}
      + D_3 \mathcal{B}_{\alpha}{}^{\mu}{}_{\nu}\mathcal{B}_{\beta}{}^{\lambda}{}_{\gamma} \nabla_\lambda \mathcal{B}_{\mu}{}^{\nu}{}_{\delta}
      + D_4 \mathcal{B}_{\alpha}{}^{\lambda}{}_{\beta}\mathcal{B}_{\gamma}{}^{\sigma}{}_{\delta}\nabla_\lambda \mathcal{A}_\sigma
      + D_5 \mathcal{B}_{\alpha}{}^{\lambda}{}_{\beta} \mathcal{A}_\sigma \nabla_\lambda \mathcal{B}_{\gamma}{}^{\sigma}{}_{\delta}
      \\
      & \quad
      + D_6 \mathcal{B}_{\alpha}{}^{\lambda}{}_{\beta}\mathcal{A}_\gamma \nabla_\lambda A_\delta
      + D_7\mathcal{B}_{\alpha}{}^{\lambda}{}_{\beta} \mathcal{A}_\lambda \nabla_\gamma A_\delta
      + E_1\nabla_\rho \mathcal{B}_{\alpha}{}^{\rho}{}_{\beta} \nabla_\sigma \mathcal{B}_{\gamma}{}^{\sigma}{}_{\delta}
      + E_2 \nabla_\rho \mathcal{B}_{\alpha}{}^{\rho}{}_{\beta} \nabla_\gamma \mathcal{A}_\delta
      \\
      &\quad
      + F_1 \mathcal{B}_{\alpha}{}^{\mu}{}_{\beta} \mathcal{B}_{\gamma}{}^{\sigma}{}_{\delta} \mathcal{B}_{\mu}{}^{\lambda}{}_{\rho} \mathcal{B}_{\sigma}{}^{\rho}{}_{\lambda}
      + F_2\mathcal{B}_{\alpha}{}^{\mu}{}_{\beta} \mathcal{B}_{\gamma}{}^{\nu}{}_{\lambda} \mathcal{B}_{\delta}{}^{\lambda}{}_{\rho} \mathcal{B}_{\mu}{}^{\rho}{}_{\nu}
      + F_3 \mathcal{B}_{\nu}{}^{\mu}{}_{\lambda} \mathcal{B}_{\mu}{}^{\nu}{}_{\alpha} \mathcal{B}_{\beta}{}^{\lambda}{}_{\gamma} \mathcal{A}_\delta
      + F_4 \mathcal{B}_{\alpha}{}^{\mu}{}_{\beta}\mathcal{B}_{\gamma}{}^{\nu}{}_{\delta}\mathcal{A}_\mu \mathcal{A}_\nu \bigg].
    \end{split}
  \end{equation*}

To couple the affine action to a scalar field, we need to introduce a kinetic term in the absence of the metric tensor. In order to do so, we build 
\textit{inverse symmetric tensor densities}, by using the \textit{dimensional analysis structure technique}

\begin{equation}
      \mathrm{g}^{\mu\nu} = \left(\alpha \nabla_\lambda \mathcal{B}_{\rho}{ }^{\mu}{ }_{\sigma} + \beta \mathcal{A}_\lambda 
      \mathcal{B}_{\rho}{ }^{\mu}{ }_{\sigma}\right)\mathrm{d}V^{\nu\lambda\rho\sigma} + \gamma \mathcal{B}_{\kappa}{ }^{\mu}{ }_{\lambda}
      \mathcal{B}_{\rho}{ }^{\nu}{ }_{\sigma}\mathrm{d}V^{\kappa\lambda\rho\sigma}
\end{equation}

Using the above expression we can define the kinetic term 

\begin{equation}
    S_{\phi} = - \int \mathrm{g}^{\mu\nu} \partial_{\mu} \phi \partial_{\nu} \phi
\end{equation}

Since we want to work on the torsion-free sector, it is worth to notice that only the terms that are linear in the torsion will
have a non-trivial contribution, $C_1$ and $C_2$. Additionally, since our connection its an \textit{equi-affine} connection, the trace of the
Riemman tensor will vanish completly. Applying the same idea the to the scalar field action, only the $\alpha$ term survive. Thus,
the effective action coupled with a scalar field is

\begin{equation*}
    \begin{split}
      S_{ef}
      & =
      \int  \mathrm{d}V^{\alpha \beta \gamma \delta} \bigg[
      C_1 \mathcal{R}_{\mu\alpha}{}^{\mu}{}_{\nu} - \alpha \partial_{\alpha}\phi \partial_{\nu}
      \bigg]\nabla_\beta \mathcal{B}_{\gamma}{}^{\nu}{}_{\delta}.
    \end{split}
  \end{equation*}

\subsection{The field equations}

The field equations are obtained using Kiwosjki's formalism and taking into account the symmetries and properties of the fundamental fields, we 
vary the action with respect to the fundamental fields. In the torsion-free limit, the field equation is

\begin{equation}
\nabla_\mu \biggl[\frac{1}{\mathcal{V}(\phi)} \left(C \partial_\alpha \phi \partial_\lambda \phi - \mathcal{R}_{\alpha\lambda}\right)\mathrm{d}V^{\mu\nu\rho\alpha}\biggr] 
+ \frac{2}{3}\nabla_\mu \biggl[ \frac{1}{\mathcal{V}(\phi)}\mathcal{R}_{\alpha\theta} \delta^{[\nu}_{\lambda}\mathrm{d}V^{\rho]\alpha\mu\theta} \biggr] = 0
\end{equation}

By multiplying the left hand side of the field equation by $\epsilon_{\nu\rho\tau\beta}$, the second term vanishes completly and the field 
equation is reduced even further to

\begin{equation}
  \nabla_{[\mu}\left(\mathcal{R}_{\nu]\gamma}\frac{1}{\mathcal{V}(\phi)}\right) 
  - C \nabla_{[\mu}\left(\partial_{\nu]} \phi \partial_\gamma \phi \frac{1}{\mathcal{V}(\phi)}\right) = 0
\end{equation}

A particular solution to the above equation is

\begin{equation}
  \mathcal{R}_{\mu\nu} - C\partial_\mu \phi \partial_\nu \phi = \Lambda \mathcal{V}(\phi) g_{\mu\nu}
\end{equation}

which can be written as

\begin{equation}
  \mathcal{R}_{\mu\nu} - \frac{1}{2}\mathcal{R}g_{\mu\nu} + \Lambda \mathcal{V}(\phi) g_{\mu\nu} = 
  C\left(\partial_\mu \phi \partial_\nu \phi - \frac{1}{2}g_{\mu\nu} (\partial \phi)^2\right)
\end{equation}

Taking the divergence $\nabla^\mu$ of the above equation leads to

\begin{equation}
  C \nabla^\mu \nabla_\mu \phi - \Lambda \mathcal{V}'(\phi) = 0
\end{equation}

where $\mathcal{V}'(\phi) = \frac{\partial \mathcal{V}(\phi)}{\partial \phi}$. The above equation is the scalar field equation, can
written as
\begin{equation}
  \nabla^\mu \nabla_\mu \phi + \beta \mathcal{V}'(\phi) = 0
\end{equation}
where $\beta = -\frac{\Lambda}{C}$.

\subsection{Cosmological ansatz}

The cosmological anstaz for the symmetric part of the connection $\Gamma$ is given by
  \begin{align}
          \Gamma_{t}{}^{t}{ }_{t} & =f(t), \quad \Gamma_{i}{ }^{t}{ }_{j}=g(t) S_{i j} \\
          \Gamma_{t}{ }^{i}{ }_{j} &= h(t) \delta^{i}_{j}, \quad \Gamma_{i}{ }^{j}{ }_{k}= \gamma_{i}{ }^{j}{ }_{k}
  \end{align}
  
The cosmological anstaz for the traceless torsion tensor $\mathcal{B}$ is given by
  \begin{align*}
      \mathcal{B}_{\theta}{ }^{r}{ }_{\varphi} & = \psi (t) r^2\sin\theta \sqrt{1 - \kappa r^2} &
      \mathcal{B}_{r}{}^{\theta}{}_{\varphi} & =\frac{\psi (t) \sin \theta}{\sqrt{1 - \kappa r^2}} \\
      \mathcal{B}_{r}{}^{\varphi}{}_{\theta} & =\frac{\psi(t)}{ \sqrt{1-\kappa r^{2}} \sin \theta}
  \end{align*}
  
The cosmological anstaz for the vectorial torsion tensor $\mathcal{A}$ is given by
  \begin{align}
      \mathcal{A}_{t} = \eta(t)
  \end{align}

Since, we have the tensor $g_{\mu\nu}$ presented as a particular solution to the field equation. Thus, we need to provied
an ansatz for this tensor compatible with the symmetries of the cosmological principle
\begin{equation}
  g_{\mu\nu} = b(t)\mathrm{d}t^2 + a(t)\left(\frac{1}{1 - \kappa r^2} \mathrm{d}r^2 + r^2 \mathrm{d}\theta^2 
  + r^2 \sin^2 \theta \mathrm{d} \varphi^2 \right)
\end{equation}

Additionally, its required that the covariant derivative of the object $g_{\mu\nu}$ must vanishes completly, from which we found that
\begin{align}
  b(t) & = b_0 & h(t) & = \frac{\dot{a}(t)}{a(t)} & g(t) & = -\frac{\dot{a}(t)a(t)}{b_0}
\end{align}

\section{Cosmological Solutions}

Searching for possible cosmological solutions, we can distinguish three types of solution

\subsection{Reduced equations}

Under the cosmological ansatz the simplies form of the components of the field equations are
\begin{align}
  0 & = 3\ddot{a}(t) + C a(t)\dot{\phi}^{2}(t) + \Lambda b_0 a(t)\mathcal{V}(\phi) \\
  0 & = \ddot{a}(t)a(t) + 2\dot{a}^2(t) - 2b_0\kappa + \Lambda b_0 a^2(t)\mathcal{V}(\phi) \\
  0 & = \ddot{\phi}(t) + 3\dot{a}(t)a(t)\dot{\phi}(t) + \beta \mathcal{V}'(\phi)
\end{align}
Combining the first two equations we obtained
\begin{equation}
  3H^2(t) = \frac{1}{2}C \dot{\phi}^2(t) - \Lambda b_0 \mathcal{V}(\phi) + \frac{3b_0\kappa}{a^2(t)}
\end{equation}
which need to be solved combined with the field equation of the scalar field. At this point we can introduce the \textit{slow-roll}
aproximation 

\subsection{Parallel equations}

\subsection{Harmonic equations}

\section{Conclusions}

\end{document}