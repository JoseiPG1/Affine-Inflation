\documentclass[10pt,a4paper]{article}
\usepackage[utf8]{inputenc}
\usepackage{amsmath}
\usepackage{amsfonts}
\usepackage{amssymb}

\usepackage{vmargin}

\usepackage{graphicx}
\graphicspath{ {./images/} }


\usepackage{slashed}

\let\stdsection\section
\renewcommand\section{\newpage\stdsection}

\title{Affine Inflation in Polynomial Affine Gravity in $3+1$ dimensions}
\author{Jose Perdiguero Garate}
\date{20/12/2022}			

\begin{document}

\tableofcontents

\section{Introduction}

\section{Polynomial Affine Gravity}

\subsection{The action}

The most general action up to boundary terms is
\begin{equation*}
    \begin{split}
      S
      & =
      \int  \mathrm{d}V^{\alpha \beta \gamma \delta} \bigg[
      B_1 \mathcal{R}_{\mu\nu}{}^{\mu}{}_{\rho}\mathcal{B}_{\alpha}{}^{\nu}{}_{\beta}\mathcal{B}_{\gamma}{}^{\rho}{}_{\delta}
      + B_2 \mathcal{R}_{\alpha\beta}{}^{\mu}{}_{\rho} \mathcal{B}_{\gamma}{}^{\nu}{}_{\delta} \mathcal{B}_{\mu}{}^{\rho}{}_{\nu}
      + B_3 \mathcal{R}_{\mu\nu}{}^{\mu}{}_{\alpha} \mathcal{B}_{\beta}{}^{\nu}{}_{\gamma} \mathcal{A}_\delta
      + B_4 \mathcal{R}_{\alpha\beta}{}^{\sigma}{}_{\rho}\mathcal{B}_{\gamma}{}^{\rho}{}_{\delta}\mathcal{A}_\sigma
      \\
      & \quad
      + B_5 \mathcal{R}_{\alpha \beta}{}^{\rho}{}_{\rho} \mathcal{B}_{\gamma}{}^{\sigma}{}_{\delta} \mathcal{A}_\sigma
      + C_1 \mathcal{R}_{\mu\alpha}{}^{\mu}{}_{\nu} \nabla_\beta \mathcal{B}_{\gamma}{}^{\nu}{}_{\delta}
      + C_2 \mathcal{R}_{\alpha\beta}{}^{\rho}{}_{\rho} \nabla_\sigma \mathcal{B}_{\gamma}{}^{\sigma}{}_{\delta}
      + D_1 \mathcal{B}_{\nu}{}^{\mu}{}_{\lambda} \mathcal{B}_{\mu}{}^{\nu}{}_{\alpha} \nabla_\beta \mathcal{R}_{\gamma}{}^{\lambda}{}_{\delta}
      \\
      & \quad
      + D_2 \mathcal{B}_{\alpha}{}^{\mu}{}_{\beta} \mathcal{B}_{\mu}{}^{\lambda}{}_{\nu} \nabla_{\lambda} \mathcal{B}_{\gamma}{}^{\nu}{}_{\delta}
      + D_3 \mathcal{B}_{\alpha}{}^{\mu}{}_{\nu}\mathcal{B}_{\beta}{}^{\lambda}{}_{\gamma} \nabla_\lambda \mathcal{B}_{\mu}{}^{\nu}{}_{\delta}
      + D_4 \mathcal{B}_{\alpha}{}^{\lambda}{}_{\beta}\mathcal{B}_{\gamma}{}^{\sigma}{}_{\delta}\nabla_\lambda \mathcal{A}_\sigma
      + D_5 \mathcal{B}_{\alpha}{}^{\lambda}{}_{\beta} \mathcal{A}_\sigma \nabla_\lambda \mathcal{B}_{\gamma}{}^{\sigma}{}_{\delta}
      \\
      & \quad
      + D_6 \mathcal{B}_{\alpha}{}^{\lambda}{}_{\beta}\mathcal{A}_\gamma \nabla_\lambda A_\delta
      + D_7\mathcal{B}_{\alpha}{}^{\lambda}{}_{\beta} \mathcal{A}_\lambda \nabla_\gamma A_\delta
      + E_1\nabla_\rho \mathcal{B}_{\alpha}{}^{\rho}{}_{\beta} \nabla_\sigma \mathcal{B}_{\gamma}{}^{\sigma}{}_{\delta}
      + E_2 \nabla_\rho \mathcal{B}_{\alpha}{}^{\rho}{}_{\beta} \nabla_\gamma \mathcal{A}_\delta
      \\
      &\quad
      + F_1 \mathcal{B}_{\alpha}{}^{\mu}{}_{\beta} \mathcal{B}_{\gamma}{}^{\sigma}{}_{\delta} \mathcal{B}_{\mu}{}^{\lambda}{}_{\rho} \mathcal{B}_{\sigma}{}^{\rho}{}_{\lambda}
      + F_2\mathcal{B}_{\alpha}{}^{\mu}{}_{\beta} \mathcal{B}_{\gamma}{}^{\nu}{}_{\lambda} \mathcal{B}_{\delta}{}^{\lambda}{}_{\rho} \mathcal{B}_{\mu}{}^{\rho}{}_{\nu}
      + F_3 \mathcal{B}_{\nu}{}^{\mu}{}_{\lambda} \mathcal{B}_{\mu}{}^{\nu}{}_{\alpha} \mathcal{B}_{\beta}{}^{\lambda}{}_{\gamma} \mathcal{A}_\delta
      + F_4 \mathcal{B}_{\alpha}{}^{\mu}{}_{\beta}\mathcal{B}_{\gamma}{}^{\nu}{}_{\delta}\mathcal{A}_\mu \mathcal{A}_\nu \bigg].
    \end{split}
  \end{equation*}

To couple the affine action to a scalar field, we need to introduce a kinetic term in the absence of the metric tensor. In order to do so, we build 
\textit{inverse symmetric tensor densities}, by using the \textit{dimensional analysis structure technique}

\begin{equation}
      \mathrm{g}^{\mu\nu} = \left(\alpha \nabla_\lambda \mathcal{B}_{\rho}{ }^{\mu}{ }_{\sigma} + \beta \mathcal{A}_\lambda 
      \mathcal{B}_{\rho}{ }^{\mu}{ }_{\sigma}\right)\mathrm{d}V^{\nu\lambda\rho\sigma} + \gamma \mathcal{B}_{\kappa}{ }^{\mu}{ }_{\lambda}
      \mathcal{B}_{\rho}{ }^{\nu}{ }_{\sigma}\mathrm{d}V^{\kappa\lambda\rho\sigma}
\end{equation}

Using the above expression we can define the kinetic term 

\begin{equation}
    S_{\phi} = - \int \mathrm{g}^{\mu\nu} \partial_{\mu} \phi \partial_{\nu} \phi
\end{equation}

Since we want to work on the torsion-free sector, it is worth to notice that only the terms that are linear in the torsion will
have a non-trivial contribution, $C_1$ and $C_2$. Additionally, since our connection its an \textit{equi-affine} connection, the trace of the
Riemman tensor will vanish completly. Applying the same idea the to the scalar field action, only the $\alpha$ term survive. Thus,
the effective action coupled with a scalar field is

\begin{equation*}
    \begin{split}
      S_{ef}
      & =
      \int  \mathrm{d}V^{\alpha \beta \gamma \delta} \bigg[
      C_1 \mathcal{R}_{\mu\alpha}{}^{\mu}{}_{\nu} - \alpha \partial_{\alpha}\phi \partial_{\nu}
      \bigg]\nabla_\beta \mathcal{B}_{\gamma}{}^{\nu}{}_{\delta}.
    \end{split}
  \end{equation*}

\subsection{Cosmological ansatz}

The cosmological anstaz for the symmetric part of the connection $\Gamma$ is given by
  \begin{align}
          \Gamma_{t}{}^{t}{ }_{t} & =f(t), \quad \Gamma_{i}{ }^{t}{ }_{j}=g(t) S_{i j} \\
          \Gamma_{t}{ }^{i}{ }_{j} &= h(t) \delta^{i}_{j}, \quad \Gamma_{i}{ }^{j}{ }_{k}= \gamma_{i}{ }^{j}{ }_{k}
  \end{align}
  
The cosmological anstaz for the traceless torsion tensor $\mathcal{B}$ is given by
  \begin{align*}
      \mathcal{B}_{\theta}{ }^{r}{ }_{\varphi} & = \psi (t) r^2\sin\theta \sqrt{1 - \kappa r^2} &
      \mathcal{B}_{r}{}^{\theta}{}_{\varphi} & =\frac{\psi (t) \sin \theta}{\sqrt{1 - \kappa r^2}} \\
      \mathcal{B}_{r}{}^{\varphi}{}_{\theta} & =\frac{\psi(t)}{ \sqrt{1-\kappa r^{2}} \sin \theta}
  \end{align*}
  
The cosmological anstaz for the vectorial torsion tensor $\mathcal{A}$ is given by
  \begin{align}
      \mathcal{A}_{t} = \eta(t)
  \end{align}


\subsection{The field equations}

The field equations are obtained using Kiwosjki's formalism and taking into account the symmetries and properties of the fundamental fields, we 
vary the action with respect to the fundamental fields. In the torsion-free limit, the field equation is

\begin{equation}
\nabla_\mu \biggl[\frac{1}{\mathcal{V}(\phi)} \left(C \partial_\alpha \phi \partial_\lambda \phi - \mathcal{R}_{\alpha\lambda}\right)\mathrm{d}V^{\mu\nu\rho\alpha}\biggr] 
+ \frac{2}{3}\nabla_\mu \biggl[ \frac{1}{\mathcal{V}(\phi)}\mathcal{R}_{\alpha\theta} \delta^{[\nu}_{\lambda}\mathrm{d}V^{\rho]\alpha\mu\theta} \biggr] = 0
\end{equation}

By multiplying the left hand side of the field equation by $\epsilon_{\nu\rho\tau\beta}$, the second term vanishes completly and the field 
equation is reduced even further to

\begin{equation}
  \nabla_{[\mu}\left(\mathcal{R}_{\nu]\gamma}\frac{1}{\mathcal{V}(\phi)}\right) 
  - C \nabla_{[\mu}\left(\partial_{\nu]} \phi \partial_\gamma \phi \frac{1}{\mathcal{V}(\phi)}\right) = 0
\end{equation}

The above field equation accept three different families of solution
\begin{enumerate}
  \item \textit{Reduced:} $\mathcal{R}_{\mu\nu} - C \partial_\mu \phi \partial_\nu \phi = 0$
  \item \textit{Parallel:} $ \nabla_{\gamma}\left(\mathcal{R}_{\mu\nu}\frac{1}{\mathcal{V}(\phi)}\right) 
  - C \nabla_{\gamma}\left(\partial_{\mu} \phi \partial_\nu \phi \frac{1}{\mathcal{V}(\phi)}\right) = 0 $
  \item \textit{Harmonic:} $ \nabla_{[\gamma}\left(\mathcal{R}_{\mu]\nu}\frac{1}{\mathcal{V}(\phi)}\right) 
  - C \nabla_{[\gamma}\left(\partial_{\mu]} \phi \partial_\nu \phi \frac{1}{\mathcal{V}(\phi)}\right) = 0 $
\end{enumerate}

A particular solution to the above equation is

\begin{equation}
  \mathcal{R}_{\mu\nu} - C\partial_\mu \phi \partial_\nu \phi = \Lambda \mathcal{V}(\phi) g_{\mu\nu}
\end{equation}

which can be written as

\begin{equation}
  \mathcal{R}_{\mu\nu} - \frac{1}{2}\mathcal{R}g_{\mu\nu} + \Lambda \mathcal{V}(\phi) g_{\mu\nu} = 
  C\left(\partial_\mu \phi \partial_\nu \phi - \frac{1}{2}g_{\mu\nu} (\partial \phi)^2\right)
\end{equation}

Taking the divergence $\nabla^mu$ of the above equation leads to

\begin{equation}
  C \nabla^\mu \nabla_\mu \phi - \Lambda \mathcal{V}(\phi) = 0
\end{equation}

which is the field equation for the scalar field.



\section{Cosmological Solutions}

\section{Conclusions}

\end{document}